\documentclass[a4paper, 11pt]{article}
\usepackage[left=2cm, top=3cm, text={17cm, 24cm}]{geometry}
\usepackage[utf8]{inputenc}
\usepackage{times}
\usepackage[czech]{babel}
\usepackage{url}
\def\UrlBreaks{\do\/\do-}
\usepackage[breaklinks]{hyperref}
\usepackage{breakurl}
\hypersetup{colorlinks = true}

\begin{document}

	\begin{titlepage}
		\begin{center}
			{\Huge\textsc{
				Vysoké učení technické v Brně\\[0.3em]}}
				{\huge\textsc{Fakulta informačních technologií
			}}
			\\
			\vspace{\stretch{0.382}}
			{\LARGE
				 Typografie a publikování -- 4. projekt\\[0.3em]}
				 {\Huge{Bibliografické citace}}
			\vspace{\stretch{0.618}}
		\end{center}
		{\Large
		    \today
			\hfill
			Vojtěch Fiala
		}
	\end{titlepage}

\section{Typografie}
\subsection{Úvod}
Typografie je nauka zabývající se písmem a rozvržením textu, aby byl přehledný a estetický. Jejím cílem je učinit text co nejčitelnějším za účelem zaručení maximálního čtenářského komfortu \cite{Maria2016}. 

Typografie se dělí na dvě části. První část, mikrotypografie, se zabývá tvorbou a vzhledem písma samotného. Druhá část, makrotypografie, se zabývá jeho rozvržením na stránce \cite{Wiki2019}. Typografii samotné se věnují mnohé knihy a periodika, což můžeme vidět například na článku \cite{Stejskalova2006}.

\uv{Typografie je pro literaturu totéž, co hudební vystoupení pro kompozici: Základní akt interpretace, plný nekonečných příležitostí pro porozumění a nepochopení.} \cite{Bringhurst1992}.


\subsection{Historie typografie}
Technický pokrok vedl v 15. století k revoluci v oblasti typografie -- Guttenbergovu vynálezu knihtisku. Za nejznámnější Guttenbergovo dílo je považována první tištěná Bible, která začala vznikat po roce 1450 \cite{Uhlirova2016}.

Další významná změna přišla až ve 20. století, kdy se typografie samotná digitalizovala. To bylo zapříčiněno příchodem nových zobrazovacích zařízení -- monitorů. Se zvyšujícím se množstvím vydávaných knih se typografie stává čím dál důležitější \cite{Jirasek2015}.



\subsection{Principy a pravidla v typografii}
Aby text dobře vypadal a dobře se četl, je potřeba dodržovat základní typografická pravidla. Neměli bychom kombinovat více druhů písem a přehánět různé formy zvýrazňování. V kratších textech bychom měli používat písmo bezpatkové, v delším textu pak písmo patkové \cite{Tejkalova2014}.

\subsection{Typografie dnes}
V dnešní době se přechází čím dál častěji z tištěných publikací na elektronické. Podle jedné britské prognózy z poloviny 90. let 20. století měly dokonce již v roce 2019 tištěné knihy sloužit pouze ke sběratelským účelům \cite{Cisar2006}. 

Velké množství původně tištěných publikací se také digitalizuje. Za pomoci speciálního softwaru jsou naskenovány strany dokumentu a uloženy do formátu PDF \cite{Stern2009}.
Čtení knihy na monitoru počítače umožňuje čtenáři dohledávat si informace či procházet reference ihned po přečtení, ale za cenu snížení pozornosti plynoucí z přítomnosti monitoru samotného. Více na téma multitaskingu viz \cite{Gomolski2006}.




\newpage
\bibliographystyle{czechiso}
\renewcommand{\refname}{Literatura}
\bibliography{proj4}

\end{document}
