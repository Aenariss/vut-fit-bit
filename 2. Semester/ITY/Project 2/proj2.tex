\documentclass[a4paper, 11pt, twocolumn]{article}
\usepackage[utf8]{inputenc}
\usepackage[IL2]{fontenc}
\usepackage[czech]{babel}
\usepackage[left=1.5cm, top=2.5cm, text={18cm, 25cm}]{geometry}
\usepackage{times}
\usepackage{amsthm, amsfonts, amsmath}
\usepackage{stackrel}
\usepackage[unicode]{hyperref}

\hyphenation{jestliže}
\hyphenation{koncového}

\setlength{\parindent}{1em}

\begin{document}
	\begin{titlepage}
		\begin{center}
			{\Huge\textsc{
				Fakulta informačních technologií\\[0.4em]
				Vysoké učení technické v Brně
			}}
			\\
			\vspace{\stretch{0.382}}
			{\LARGE
				 Typografie a publikování -- 2. projekt\\[0.3em]
				 Sazba dokumentů a matematických výrazů
			}
			\vspace{\stretch{0.618}}
		\end{center}
		{\Large
		    2020
			\hfill
			Vojtěch Fiala (xfiala61)
		}
	\end{titlepage}
	
    \section*{Úvod}
    \label{page_1}
       V této úloze si vyzkoušíme sazbu titulní strany, matematic\-kých vzorců, prostředí a dalších textových struktur obvyk\-lých pro technicky zaměřené texty (například rovnice (\ref{eq:2})
        nebo Definice \ref{definice_2} na straně \pageref{page_1}). Pro vytvoření těchto odkazů
        používáme příkazy \verb!\label!, \verb!\ref! a \verb!\pageref!.
    
        Na titulní straně je využito sázení nadpisu podle op\-tického středu s využitím zlatého řezu. Tento postup byl
        probírán na přednášce. Dále je použito odřádkování se
        zadanou relativní velikostí 0.4em a 0.3em.
    
    \section{Matematický text}
    
        Nejprve se podíváme na sázení matematických symbolů
        a~výrazů v plynulém textu včetně sazby definic a vět s vy\-užitím balíku \verb!amsthm!. Rovněž použijeme poznámku pod
        čarou s použitím příkazu \verb!\footnote!. Někdy je vhodné
        použít konstrukci \verb!${}$! nebo \verb!\mbox{}! která říká, že
        (matematický) text nemá být zalomen. V následující de\-finici je nastavena mezera mezi jednotlivými položkami
        \verb!\item! na 0.05em.
        
    \newtheorem{Def1}{Definice}
    \begin{Def1}
    \label{definice_1}
        {\textnormal{Turingův stroj}} (TS) je definován jako šestice
        tvaru $M = (Q, \Sigma, \Gamma, \delta, q_0, q_F$), kde:
        
        \begin{itemize}
            \setlength\itemsep{0.05em}
            \item $Q$ je konečná množina {\textnormal{vnitřních (řídicích) stavů}},
            \item $\Sigma$ je konečná množina symbolů nazývaná  {\textnormal{vstupní
                abeceda}}, $\Delta \notin \Sigma$,
            \item $\Gamma$ je konečná množina symbolů, $\Sigma \subset \Gamma$, $\Delta \in \Gamma$,
                nazývaná {\textnormal{pásková abeceda}},
            \item $\delta$ {\textnormal{:}} $(Q \backslash \{q_F\})\times\Gamma$ $\rightarrow Q\times(\Gamma\cup\{L, R\})$, kde $L, R \notin \Gamma$, je parciální {\textnormal{přechodová funkce}}, a
            \item  $q_0 \in Q$ je {\textnormal{počáteční stav}} a $q_f \in Q$ je {\textnormal{koncový stav}}.
        \end{itemize}
    \end{Def1}
    
        Symbol $\Delta$ značí tzv. \emph{blank} (prázdný symbol), který se
        vyskytuje na místech pásky, která nebyla ještě použita.
    
        \emph{Konfigurace pásky} se skládá z nekonečného řetězce,
        který reprezentuje obsah pásky a pozice hlavy na tomto
        řetězci. Jedná se o prvek množiny $\{\gamma\Delta^{\omega}$
        $\vert$  $\gamma \in \Gamma^{\ast}\} \times \mathbb{N}$\footnote{Pro libovolnou abecedu $\Sigma$ je $\Sigma^{\omega}$ množina všech nekonečných řetězců nad $\Sigma$, tj. nekonečných posloupností symbolů ze $\Sigma$.}.
        \emph{Konfiguraci pásky} obvykle zapisujeme jako $\Delta xyz\underline{z}x\Delta$...
        (podtržení značí pozici hlavy). \emph{Konfigurace stroje} je pak
        dána stavem řízení a konfigurací pásky. Formálně se jedná
        o prvek množiny $Q \times \{\gamma\Delta^{\omega}$ $\vert$ $\gamma \in \Gamma^{\ast}\} \times \mathbb{N}$.
        
    \subsection{Podsekce obsahující větu a odkaz}
    \newtheorem{Def2}[Def1]{Definice}
    \begin{Def2}
    \label{definice_2}
        {\textnormal{Řetězec}} $w$ {\textnormal{nad abecedou}} $\Sigma$ {\textnormal{je přijat TS}} $M$ jestliže $M$ při aktivaci z počáteční konfigurace pásky\break $\underline{\Delta}w\Delta$... a počátečního stavu $q_0$ zastaví přechodem do koncového stavu $q_F$, tj. $(q_0, \Delta w\Delta^{\omega},0)$ ${\stackrel[M]{\ast}{\vdash}}$ $(q_F, \gamma, n)$ pro nějaké $\gamma \in \Gamma^{\ast}$ a $n \in  \mathbb{N}$.
        
        Množinu $L(M) = \{w$ $\vert$ $w$ je přijat TS $M\}$ $\subseteq$ $\Sigma^{\ast}$ nazýváme {\textnormal{jazyk přijímaný TS}} $M$.
    \end{Def2}
    Nyní si vyzkoušíme sazbu vět a důkazů opět s použitím
    balíku \texttt{amsthm}.
    
    \newtheorem{Veta}{Věta}
    \begin{Veta}
        Třída jazyků, které jsou přijímány TS, odpovídá
        {\textnormal{rekurzivně vyčíslitelným jazykům}}.
    \end{Veta}
    \begin{proof}[Důkaz]
        V důkaze vyjdeme z Definice \ref{definice_1} a \ref{definice_2}.
    \end{proof}
    
    \section{Rovnice}
        Složitější matematické formulace sázíme mimo plynulý
        text. Lze umístit několik výrazů na jeden řádek, ale pak je
        třeba tyto vhodně oddělit, například příkazem \verb!\quad!.
        
        $$
            \sqrt[i]{x^{3}_{i}} 
            \quad
            \textnormal{kde $x_i$ je $i$-té sudé číslo}
            \quad
            y^{2\cdot y_i}_i \neq y^{y^{y_i}_i}_i
        $$
        
        V rovnici (\ref{eq:1}) jsou využity tři typy závorek s různou
        explicitně definovanou velikostí.
        
        \begin{align}
            \begin{split}\label{eq:1}
            x\quad={}&\quad \bigg\{\Big(\big[a + b\big]\ast c\Big)^{d}\oplus 1\bigg\}
            \end{split}\\
            \begin{split}\label{eq:2}
            y\quad={}&\quad \lim\limits_{x\to\infty}\frac{\sin^{2}x + \cos^{2}x}{\frac{1}{\log_{10}{x}}}
            \end{split}
        \end{align}
        
        V této větě vidíme, jak vypadá implicitní vysázení limity $\lim_{n\to\infty} f(n)$ v normálním odstavci textu. Podobně je to i s dalšími symboly jako $\sum_{i=1}^{n} 2^{i}$ či $\bigcap_{A\in \mathcal{B}} A$. V~pří\-padě vzorců $\lim\limits_{n\to\infty} f(n)$ a $\sum\limits_{i=1}^{n} 2^{i}$ jsme si vynutili méně úspornou sazbu příkazem \verb!\limits!.
        
    \section{Matice}
        Pro sázení matic se velmi často používá prostředí \verb!array! a závorky (\verb!\left!, \verb!\right!).
        
        $$ \left(
        \begin{array}{ccc}
             a+b & \widehat{\xi + \omega} & \hat{\pi} 
             \\
            \vec{\textbf{a}} & \overleftrightarrow{AC} & \beta
        \end{array}
        \right) = 1  \Longleftrightarrow \mathbb{Q} = \mathcal{R}$$
        Prostředí \verb!array! lze úspěšně využít i jinde.
        
        $$\binom{n}{k}
        = 
        \left\{ \begin{array}{cl}
             0 & \textnormal{pro $k < 0$ nebo $k > n$}\\
             \frac{n!}{k!(n-k)!} & \textnormal{pro $0 \leq k \leq n$.}
        \end{array}\right.
        $$


        
\end{document}
